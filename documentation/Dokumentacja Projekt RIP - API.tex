\documentclass{article}
\usepackage{polski}
\usepackage[utf8]{inputenc}
\usepackage{hyperref}

\hypersetup{
    colorlinks=true,
    linkcolor=blue,
    filecolor=magenta,      
    urlcolor=cyan,
    pdfpagemode=FullScreen,
    }

\title{Projekt\textunderscore RIP - API}
\author{
Jakub Engielski\\
Jan Kwiatkowski \\
}


\date{Styczeń 2023}

\usepackage{natbib}
\usepackage{graphicx}

\begin{document}

\maketitle

\section{Opis funkcjonalny systemu}
Celem "Projektu\textunderscore RIP - API" było utworzenie aplikacji do pobrania danych z planu uczelni Collegium Witelona. Wraz z tym dochodzi także część formatująca, przystosowująca pobrane dane w ten sposób aby aplikacja web oraz aplikacja mobilna byłby w stanie bez problemu się nimi obsługiwać.
\\Api będzie zgodnie z ustalonym harmonogramem, np. raz dziennie pobierało dane z planu uczelni. Wykonywało swoje funkcje, oraz wystawiało już sformatowane i posegregowane dane przy pomocy rest api.
\\Endpointy wykonane są zgodnie z standardami swaggera. 


\section{Streszczenie opisu technologicznego}
Korzystamy z framework'u Symfony 6.1 wspierającego język PHP, dzięki któremy jesteśmy w stanie postawić ten projekt bez potrzeby pisania wielu rzeczy od nowa. Wybraliśmy Symfony gdyż jesteśmy z nim najbardziej zapoznani. \\
Docker umożliwia nam łatwe pakowanie, dostarczanie oraz uruchamianie aplikacji w formie lekkiego, przenośnego i samowystarczalnego kontenera. Co pozwala nam to uruchomić szybko i praktycznie wszędzie.
\\
Api Platform jest paczką, która pozwala na wygodne i szybkie wystawienie rest api. Bez tego prace zostały by wydłużone bez żadnego dobrego powodu. 
\\
Goutte jest paczką pozwalającą nam na postawienie samego scrapera. Jest to niezbędna część tego projektu, bez niej postawienie scrapera w php byłoby o wiele bardziej obciażające jeśli chodzi o moc obliczeniową (Symfony Panther) albo wymagało by olbrzymiej pracy na około aby obejść prostotę systemu (Guzzle)
\\
Doctrine jest paczką pozwalającą nam w wygodnu sposób pracować nad entity. Dzięki czemu jesteśmy w stanie wyprowadzić poprawne dane w formie api.
\\
Bazą danych jaka jest wykorzystawna jest Postgres. Jest to zaawansowany system do zarządania relacyjnymi bazami danych, do tego jest to projekt open source.

\section{Instrukcję lokalnego i zdalnego uruchomienia systemu}
\subsection{Postawienie systemu lokalnie}
Wymagane oprogramowanie:\\
PHPStorm lub dowolny inny IDE\\
Docker Desktop\\
Terminal Windows (nie jest potrzebny osobno jeśli jest wbudowany w IDE jak w PHPStorm)\\
Github Desktop – aby móc wprowadzać zmiany, bądź pobierać aktualizacje jeśli są potrzebne. (nie jest wymagany  jeśli jest wbudowany w IDE, bądź jeśli ktoś posiada zainstalowany pakiet GIT do użytku poprzez terminal)\\\\
Jak postawić środowisko testowe na dockerze?\\
\href{https://docs.docker.com/docker-for-windows/}{Instrukcja dla Windowsa}\\
\href{https://docs.docker.com/compose/install/}{Instrukcja dla Linuxa}\\\\
\textbf{Jeżeli wszystko zainstalowałeś, przejdź do instrukcji poniżej:}
\begin{enumerate}
    \item Pobranie projektu z repozytorium oraz przejście do jego folderu: \\\\
    \emph{ git clone https://github.com/sbacanski0730/RIP-Rewak-and-PUM-API.git} \\
   			 \emph{ cd RIP-Rewak-and-PUM-API} \\
    \item Konfiguracja pliku .env\\\\
    \emph{cp .env .env.local} \\
    \item Uruchomienie kontenerów dockerowych:\\\\
    \emph{docker-compose up -d --build}\\
    \item Pobranie zależności Composer:\\\\
    \emph{docker-compose exec php composer install}\\
 			 \item Pobranie zależności npm:\\\\
 		   \emph{docker-compose exec php npm install}\\
 			 \item Budowanie Assetów:\\\\
 		   \emph{docker-compose exec php npm run dev}\\
    \end{enumerate}
\textbf{Migracje:}\\
\begin{enumerate}
\item Utworzenie migracji: \\\\
 \emph{php bin/console doctrine:migrations:diff} \\
 \emph{php bin/console doctrine:migrations:migrate} \\\\
  \end{enumerate}
 \textbf{Bez migracji:}\\
 \begin{enumerate}
\item Utworzenie schematu bazy danych: \\\\
 \emph{php bin/console doctrine:schema:update --force} \\
 \item Załadowanie zdefiniowanych wczesniej danych do bazy danych: \\\\
 \emph{php bin/console doctrine:fixtures:load} \\
 \end{enumerate}






\textbf{Usługi:}
    \begin{itemize}
        \item api https://localhost:443/api
        \item postgres localhost:5432 \\\\
    \end{itemize}
    
    
   
\textbf{Przydatne Komendy:}\\
\begin{enumerate}
\item Docker - uruchomienie kontenerów: \\\\
 \emph{docker-compose up -d} \\
 \item Docker - zatrzymanie kontenerów: \\\\
 \emph{docker-compose stop} \\
 \item Docker - zatrzymanie i usunięcie kontenerów \\\\
 \emph{docker-compose down} \\
  \end{enumerate}
    

\subsection{Postawienie systemu zdalnie}
To Edit:
W naszym wypadku użyliśmy programu PuTTY, aby podłączyć się z serwerem Debianowym. Może do tego posłużyć dowolny klient SSH.\\
Po tym musieliśmy zainstalować następujące pakiety:
\begin{itemize}
\item nginx 1.18
\item PHP 8.0.14
\item mariadb 10.5.12
\item phpMyAdmin 5.0.4
\item Node.js 16.13.2
\end{itemize}
Pliki naszej strony są automatycznie pobierane z repozytorium git.
\begin{enumerate}
\item Pobieranie zależności Composer'a: \\\\
 \emph{composer install
} \\
 \item Pobranie zależności npm:\\\\
 \emph{npm install
} \\
 \item Budowanie Assetów: \\\\
 \emph{yarn build
} \\
 \item Utworzenie schematu bazy danych: \\\\
 \emph{php bin/console doctrine:schema:update --force
} \\
 \item Załadowanie zdefiniowanych wczesniej danych do bazy danych: \\\\
 \emph{php bin/console doctrine:fixtures:load
} \\

  \end{enumerate}


\section{Dokumentacja}
Link do pełnej dokumentacji na naszym repozytorium: 
api: \url{https://github.com/sbacanski0730/RIP-Rewak-and-PUM-API/tree/main/documentation}\\\\
web:
mobile:

\section{Wnioski projektowe}



\end{document}